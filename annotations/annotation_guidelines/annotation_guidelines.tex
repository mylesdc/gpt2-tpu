%%%%%%%%%%%%%%%%%%%%%%%%%%%%%%%%%%%%%%%%%%%%%%%%%%%%%%%%%%%%%%%%%%%%%%%%%%%%%%%
%
% Filename: annotation_guidelines.tex
% Author:   David Oniani
% Modified: April 02, 2020
%  _         _____   __  __
% | |    __ |_   _|__\ \/ /
% | |   / _` || |/ _ \\  /
% | |__| (_| || |  __//  \
% |_____\__,_||_|\___/_/\_\
%
%%%%%%%%%%%%%%%%%%%%%%%%%%%%%%%%%%%%%%%%%%%%%%%%%%%%%%%%%%%%%%%%%%%%%%%%%%%%%%%

%%%%%%%%%%%%%%%%%%%%%%%%%%%%%%%%%%%%%%%%%%%%%%%%%%%%%%%%%%%%%%%%%%%%%%%%%%%%%%%
% Document Definition
%%%%%%%%%%%%%%%%%%%%%%%%%%%%%%%%%%%%%%%%%%%%%%%%%%%%%%%%%%%%%%%%%%%%%%%%%%%%%%%

\documentclass[11pt]{article}

%%%%%%%%%%%%%%%%%%%%%%%%%%%%%%%%%%%%%%%%%%%%%%%%%%%%%%%%%%%%%%%%%%%%%%%%%%%%%%%
% Packages and Related Settings
%%%%%%%%%%%%%%%%%%%%%%%%%%%%%%%%%%%%%%%%%%%%%%%%%%%%%%%%%%%%%%%%%%%%%%%%%%%%%%%

% Global, document-wide settings
\usepackage[margin=1in]{geometry}
\usepackage[utf8]{inputenc}
\usepackage[english]{babel}

% Other packages
\usepackage{booktabs}
\usepackage{fancyhdr}
\usepackage{hyperref}
\usepackage[cache=false]{minted}
\usepackage{tocloft}

%%%%%%%%%%%%%%%%%%%%%%%%%%%%%%%%%%%%%%%%%%%%%%%%%%%%%%%%%%%%%%%%%%%%%%%%%%%%%%%
% Command Definitions and Redefinitions
%%%%%%%%%%%%%%%%%%%%%%%%%%%%%%%%%%%%%%%%%%%%%%%%%%%%%%%%%%%%%%%%%%%%%%%%%%%%%%%

% New commands
\newcommand{\rarr}{\rightarrow}             % Leftarrow
\newcommand{\larr}{\leftarrow}              % Rightarrow
\newcommand\und[1]{\underline{\smash{#1}}}  % Nice-looking underline
\renewcommand{\baselinestretch}{1.5}        % Line spacing is 1.5

% Rename "Contents" to "Table of Contents"
\addto\captionsenglish{% Replace "english" with the language used
  \renewcommand{\contentsname}%
    {\textbf{Table of Contents}}}%

% Filling the space for centering the title of the table of contents
% Dots for ToC sections
\renewcommand{\cftsecleader}{\cftdotfill{\cftdotsep}}
\renewcommand{\cfttoctitlefont}{\hspace*{\fill}\Large}
\renewcommand{\cftaftertoctitle}{\hspace*{\fill}}

%%%%%%%%%%%%%%%%%%%%%%%%%%%%%%%%%%%%%%%%%%%%%%%%%%%%%%%%%%%%%%%%%%%%%%%%%%%%%%%
% Miscellaneous
%%%%%%%%%%%%%%%%%%%%%%%%%%%%%%%%%%%%%%%%%%%%%%%%%%%%%%%%%%%%%%%%%%%%%%%%%%%%%%%

% Setting stuff
\setlength{\parindent}{0pt}  % Remove indentations from paragraphs
\pagestyle{fancy}            % This allows to do fancy headers and footers
\fancyhf{}                   % No additional page numbering (or other stuff)
\cfoot{\thepage}             % Display page number at the bottom, in the center

% PDF information and nice-looking urls
\hypersetup{%
  pdfauthor={David Oniani},
  pdftitle={Annotation Guidelines for COVID-19 Chatbot},
  pdfsubject={COVID-19, Aritificial Intelligence, Mayo Clinic},
  pdfkeywords={COVID-19, Aritificial Intelligence, Mayo Clinic},
  pdflang={English},
  colorlinks=true,
  linkcolor={black!50!blue},
  citecolor={black!50!blue},
  urlcolor={black!50!blue}
}

% Put a centered header of a footnote size on the top of each page
\lhead{\footnotesize{Annotation Guidelines for COVID-19 Chatbot}}
\rhead{\footnotesize{David Oniani and Dr.~Yanshan Wang}}

%%%%%%%%%%%%%%%%%%%%%%%%%%%%%%%%%%%%%%%%%%%%%%%%%%%%%%%%%%%%%%%%%%%%%%%%%%%%%%%
% Author(s), Title, and Date
%%%%%%%%%%%%%%%%%%%%%%%%%%%%%%%%%%%%%%%%%%%%%%%%%%%%%%%%%%%%%%%%%%%%%%%%%%%%%%%

% Author(s)
\author{David Oniani\\
        Mayo Clinic\\
        \href{mailto:oniani.david@mayo.edu}{mailto:oniani.david@mayo.edu}
        \and
        Dr.~Yanshan Wang\\
        Mayo Clinic\\
        \href{mailto:wang.yanshan@mayo.edu}{mailto:wang.yanshan@mayo.edu}}

% Title
\title{\textbf{Annotation Guidelines for COVID-19 Chatbot}}

% Date
\date{\today}

%%%%%%%%%%%%%%%%%%%%%%%%%%%%%%%%%%%%%%%%%%%%%%%%%%%%%%%%%%%%%%%%%%%%%%%%%%%%%%%
% Beginning of the Document
%%%%%%%%%%%%%%%%%%%%%%%%%%%%%%%%%%%%%%%%%%%%%%%%%%%%%%%%%%%%%%%%%%%%%%%%%%%%%%%

\begin{document}
\maketitle

%%%%%%%%%%%%%%%%%%%%%%%%%%%%%%%%%%%%%%%%%%%%%%%%%%%%%%%%%%%%%%%%%%%%%%%%%%%%%%%
% Table of Contents
%%%%%%%%%%%%%%%%%%%%%%%%%%%%%%%%%%%%%%%%%%%%%%%%%%%%%%%%%%%%%%%%%%%%%%%%%%%%%%%

\tableofcontents
\newpage

%%%%%%%%%%%%%%%%%%%%%%%%%%%%%%%%%%%%%%%%%%%%%%%%%%%%%%%%%%%%%%%%%%%%%%%%%%%%%%%
% Question Categorization
%%%%%%%%%%%%%%%%%%%%%%%%%%%%%%%%%%%%%%%%%%%%%%%%%%%%%%%%%%%%%%%%%%%%%%%%%%%%%%%

\section{Answer Categorization}

Answers will be categorized according to 5 categories:

\begin{itemize}
  \item	Relevant -- the answer partially or fully answers the question and/or
        makes clear attempts to do so and is related to the question.
        (5 points)

  \item	Well-formed -- the answer makes a logical sense and is somewhat related
        to both the question and COVID-19, yet it does not (partially or fully)
        answer the question.
        (4 points)

  \item	Informative -- the answer is not related to the question, but provides
        some information about COVID-19 and makes a logical sense.
        (3 points)

  \item	Acceptable -- the answer makes some logical sense and is weakly related
        to the question or COVID-19, but is mostly difficult to understand.
        (2 points)

  \item	Poor -- the answer is totally unrelated to the question or COVID-19
        and/or does not make a logical sense.
        (1 point)
\end{itemize}

%%%%%%%%%%%%%%%%%%%%%%%%%%%%%%%%%%%%%%%%%%%%%%%%%%%%%%%%%%%%%%%%%%%%%%%%%%%%%%%
% Example Answers and Evaluation
%%%%%%%%%%%%%%%%%%%%%%%%%%%%%%%%%%%%%%%%%%%%%%%%%%%%%%%%%%%%%%%%%%%%%%%%%%%%%%%

\section{Example Answers and Evaluation}

The answers below are made-up and are not present in the actual evaluation
data.

\subsection{Relevant (5 Points)}

Question: Are there geographic variations in the mortality rate of COVID-19?

\textbf{Answer}: \textit{The mortality rate of COVID-19 depends on the
geography a lot and some places in the world might be more vulnerable to it
than others. It is also important to consider other causes. I love ice cream!}

\medskip

Explanation: The answer would be rated 5 since it is clearly related to the
question. Although the answer mentions unrelated bits like ``I love ice
cream'', it is mostly relevant to the question.

\subsection{Well-formed (4 Points)}

Question: Are there geographic variations in the mortality rate of COVID-19?

\textbf{Answer}: \textit{Geography is a great subject! You should learn it!
Also, COVID-19 has been tough geographically. It is terrible if you get it.}

\medskip

Explanation: The answer is somewhat related to the question since it mentions
both the geography and COVID-19, however, it does not answer the question.

\subsection{Informative (3 Points)}

Question: Are there geographic variations in the mortality rate of COVID-19?

\textbf{Answer}: \textit{COVID-19 or Novel coronavirus is a threat to human
health. It is bad if you get it so be safe. Some scientists think that it
spreads faster than Influenza.}

\medskip

Explanation: Although the text is not directly related the question, it
provides some information about COVID-19 and makes a logical sense.

\subsection{Acceptable (2 Point)}

Question: Are there geographic variations in the mortality rate of COVID-19?

\textbf{Answer}: \textit{Some disease are viral. There has been viral diseases
recently. It can be damaging to people.}

\medskip

Explanation: The answer is weakly related to COVID-19 since it mentions terms
``disease'' and ``viral.'' Yet, it fails to deliver the answer related to
COVID-19 or the question.

\subsection{Poor (1 Point)}

Question: Are there geographic variations in the mortality rate of COVID-19?

\textbf{Answer}: \textit{To be, or not to be, that is the question???! What is
life? What is Earth? What is human? Soma animals are carnivores and some are
herbivores, and some omnivores.}

\medskip

Explanation: The answer is totally unrelated to the question or COVID-19.

%%%%%%%%%%%%%%%%%%%%%%%%%%%%%%%%%%%%%%%%%%%%%%%%%%%%%%%%%%%%%%%%%%%%%%%%%%%%%%%
% The End of the Document
%%%%%%%%%%%%%%%%%%%%%%%%%%%%%%%%%%%%%%%%%%%%%%%%%%%%%%%%%%%%%%%%%%%%%%%%%%%%%%%

\end{document}
